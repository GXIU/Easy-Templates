%------------------------------------------------
\subsection{研究现状}

\begin{frame}{研究现状}
多稳定形态流行病状态的理解
    \begin{itemize}
        \item 流行病的时空动态研究
        \item 基于网络的流行病传播研究
        \item 流行病传播背景下的社会生产问题
        \item ……
    \end{itemize}
\pause
基于(时空)去中心化思想的控制策略研究
\begin{itemize}
    \item 移动性假定下的流行病传播阈值
    \item 人类移动性与基于媒介的流行病
    \item ……
\end{itemize}
\end{frame}

\begin{frame}{流行病的时空传播规律}
    时空传播规律与医疗系统设计和对新疫情出现时的理解息息相关。流行病的具体传播规律,不光是参数的设定,更是大类模型的选择。要关注非线性、时空异质性、非马氏性等重要因素。
    \begin{itemize}
        \item Travelling waves and spatial hierarchies in measles epidemics, Bryan Grenfell et. al., Nature 2001
        \item Power laws governing epidemics in isolated populations, C. J. Rhodes and R. M. Anderson
        \begin{itemize}
            \item 与稀树草原-森林演化的林火模型之间的关系
        \end{itemize}
        \item Synchrony, Waves, and Spatial Hierarchies in the Spread of Influenza, Science 2006
        \begin{itemize}
            \item 小波相位分析,可以实现动态非平稳性的探测
        \end{itemize}
        \item From individuals to epidemics
    \end{itemize}
\end{frame}

\begin{frame}{网络模型}
城市本身的社交属性使得城市中的移动性模式、个体选择呈现着很高的异质性。不同国家对防疫策略的反应与网络刻画的社交关系是高度相关的。
\vspace{0.5cm}

城市复工政策选择方面,公司的空间区位与相关关系也是考虑的重点。该关系在生态学理论中是以网络来刻画的。
    \begin{itemize}
        \item The influence of awareness on epidemic spreading on random networks, Volume 486, 7 February 2020, 110090, Journal of Theoretical Biology, 
        \item Impacts of preference and geography on epidemic spreading, Xin-Jian Xu, et. al. Phys. Rev. E 76, 056109 2007
        \item Epidemic thresholds in dynamic contact networks, J. R. Soc. Interface (2009) 6, 233–241, Erik Volz, and Lauren Ancel Meyers
    \end{itemize}
\end{frame}
%------------------------------------------------
\subsection{问题归纳}

\begin{frame}{问题归纳}
    \begin{enumerate}
        \item 停工/复工策略
        \item 舆论场背景下,防疫措施与
        \item 移动性限制与疾病最优阈值
    \end{enumerate}
\end{frame}