\begin{frame}{选题出发点}
    2020年,自中国起,全球人民为抗击COVID-19病毒肺炎疫情付出了巨大的代价。城市作为人类文明的集中载体、社会性生态的温室,在流行病、战争、能源/粮食危机面前,则会反映出一定的脆弱性。可以预见的是,关于城市效率与韧性的相互制衡将极大影响规划者的长期设计。
    
    \vspace{0.3cm}
    
    在信息时代,城市中社交关系、移动性模式处于演化过程中。针对不同的时期的技术与社会环境,有着不同的较优解。与此同时,在生态学、复杂系统研究、控制论等领域出现了很多理解与调控系统结构以优化系统功能的研究。因此,城市系统精细结构的研究有着较大的余地与需求。
    
    \vspace{0.3cm}
    
    本文将以流行病的在城市空间中的传播作为切入点,针对城市生态系统的韧性和功能作出一系列讨论。以期在实践上给出突发影响时期城市人群应作的调整;在理论上提出城市中社交关系、移动性模式等交互模式所产生的全局影响的新解释。
\end{frame}{}
\subsection{城市生态与城市韧性}
\begin{frame}{城市生态与城市韧性}
    \begin{itemize}
    \item 城市的物质、信息、人口等要素在空间上的异质性交互使得城市体现出了复杂的跨尺度、临界态、多重耦合的生态性特征。
    \item 城市网络作为一种新的城市地域系统组织形式和研究范式被广泛关注。目前绝大多数城市网络研究都基于正面视角,对城市网络的负面效应、安全与可持续发展问题却鲜有提及。\cite{Xiu2019}
    \item 从近代历史来看,高人口密度、高物质交互流量、高耦合性的现代城市生态堪称流行病的温床。
\end{itemize}
\end{frame}

\begin{frame}{流行病控制措施}
    \begin{itemize}
        \item 2020年新冠肺炎时期,武汉等城市采取了严厉的封城措施,极大减轻了市际传染程度
        \item 娱乐场所等的暂停营业减少了流行病传播的潜在场所
        \item 时间、空间两个层面的“去中心化”,减轻医疗系统负担,“群体免疫”策略
        \item 有些国家针对佩戴口罩、居家隔离等措施产生了社会大讨论,甚至聚众游行集会等疾控意义上的灾难性事件
    \end{itemize}
    \vspace{0.5cm}
    \pause
    \begin{center}
    \large{
        \textit{移动性、恢复生产、舆论环境 应是流行病控制的主要考量}}
    \end{center}
\end{frame}

\begin{frame}{现有研究及广泛意义}
    \begin{itemize}
        \item \textbf{数学模型}
        \begin{itemize}
            \item 基于SEIR等模型的元人口/网络模型研究
        \end{itemize}
        \vspace{0.5cm}
        \item \textbf{城市动态演化}\begin{itemize}
            \item 长时间尺度上流行病的时空动态与城市模式的关系
        \end{itemize}
        \vspace{0.5cm}
        \item \textbf{效率与韧性,城市的鸡尾酒瓶}\begin{itemize}
            \item 控制论、复杂网络理论给生态系统结构及其鲁棒性的研究提供了强有力的工具
        \end{itemize}
    \end{itemize}
\end{frame}